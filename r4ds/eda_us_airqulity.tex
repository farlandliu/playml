\documentclass[]{article}
\usepackage{lmodern}
\usepackage{amssymb,amsmath}
\usepackage{ifxetex,ifluatex}
\usepackage{fixltx2e} % provides \textsubscript
\ifnum 0\ifxetex 1\fi\ifluatex 1\fi=0 % if pdftex
  \usepackage[T1]{fontenc}
  \usepackage[utf8]{inputenc}
\else % if luatex or xelatex
  \ifxetex
    \usepackage{mathspec}
  \else
    \usepackage{fontspec}
  \fi
  \defaultfontfeatures{Ligatures=TeX,Scale=MatchLowercase}
\fi
% use upquote if available, for straight quotes in verbatim environments
\IfFileExists{upquote.sty}{\usepackage{upquote}}{}
% use microtype if available
\IfFileExists{microtype.sty}{%
\usepackage{microtype}
\UseMicrotypeSet[protrusion]{basicmath} % disable protrusion for tt fonts
}{}
\usepackage[margin=1in]{geometry}
\usepackage{hyperref}
\hypersetup{unicode=true,
            pdftitle={EDA of us airquality},
            pdfborder={0 0 0},
            breaklinks=true}
\urlstyle{same}  % don't use monospace font for urls
\usepackage{color}
\usepackage{fancyvrb}
\newcommand{\VerbBar}{|}
\newcommand{\VERB}{\Verb[commandchars=\\\{\}]}
\DefineVerbatimEnvironment{Highlighting}{Verbatim}{commandchars=\\\{\}}
% Add ',fontsize=\small' for more characters per line
\usepackage{framed}
\definecolor{shadecolor}{RGB}{248,248,248}
\newenvironment{Shaded}{\begin{snugshade}}{\end{snugshade}}
\newcommand{\KeywordTok}[1]{\textcolor[rgb]{0.13,0.29,0.53}{\textbf{#1}}}
\newcommand{\DataTypeTok}[1]{\textcolor[rgb]{0.13,0.29,0.53}{#1}}
\newcommand{\DecValTok}[1]{\textcolor[rgb]{0.00,0.00,0.81}{#1}}
\newcommand{\BaseNTok}[1]{\textcolor[rgb]{0.00,0.00,0.81}{#1}}
\newcommand{\FloatTok}[1]{\textcolor[rgb]{0.00,0.00,0.81}{#1}}
\newcommand{\ConstantTok}[1]{\textcolor[rgb]{0.00,0.00,0.00}{#1}}
\newcommand{\CharTok}[1]{\textcolor[rgb]{0.31,0.60,0.02}{#1}}
\newcommand{\SpecialCharTok}[1]{\textcolor[rgb]{0.00,0.00,0.00}{#1}}
\newcommand{\StringTok}[1]{\textcolor[rgb]{0.31,0.60,0.02}{#1}}
\newcommand{\VerbatimStringTok}[1]{\textcolor[rgb]{0.31,0.60,0.02}{#1}}
\newcommand{\SpecialStringTok}[1]{\textcolor[rgb]{0.31,0.60,0.02}{#1}}
\newcommand{\ImportTok}[1]{#1}
\newcommand{\CommentTok}[1]{\textcolor[rgb]{0.56,0.35,0.01}{\textit{#1}}}
\newcommand{\DocumentationTok}[1]{\textcolor[rgb]{0.56,0.35,0.01}{\textbf{\textit{#1}}}}
\newcommand{\AnnotationTok}[1]{\textcolor[rgb]{0.56,0.35,0.01}{\textbf{\textit{#1}}}}
\newcommand{\CommentVarTok}[1]{\textcolor[rgb]{0.56,0.35,0.01}{\textbf{\textit{#1}}}}
\newcommand{\OtherTok}[1]{\textcolor[rgb]{0.56,0.35,0.01}{#1}}
\newcommand{\FunctionTok}[1]{\textcolor[rgb]{0.00,0.00,0.00}{#1}}
\newcommand{\VariableTok}[1]{\textcolor[rgb]{0.00,0.00,0.00}{#1}}
\newcommand{\ControlFlowTok}[1]{\textcolor[rgb]{0.13,0.29,0.53}{\textbf{#1}}}
\newcommand{\OperatorTok}[1]{\textcolor[rgb]{0.81,0.36,0.00}{\textbf{#1}}}
\newcommand{\BuiltInTok}[1]{#1}
\newcommand{\ExtensionTok}[1]{#1}
\newcommand{\PreprocessorTok}[1]{\textcolor[rgb]{0.56,0.35,0.01}{\textit{#1}}}
\newcommand{\AttributeTok}[1]{\textcolor[rgb]{0.77,0.63,0.00}{#1}}
\newcommand{\RegionMarkerTok}[1]{#1}
\newcommand{\InformationTok}[1]{\textcolor[rgb]{0.56,0.35,0.01}{\textbf{\textit{#1}}}}
\newcommand{\WarningTok}[1]{\textcolor[rgb]{0.56,0.35,0.01}{\textbf{\textit{#1}}}}
\newcommand{\AlertTok}[1]{\textcolor[rgb]{0.94,0.16,0.16}{#1}}
\newcommand{\ErrorTok}[1]{\textcolor[rgb]{0.64,0.00,0.00}{\textbf{#1}}}
\newcommand{\NormalTok}[1]{#1}
\usepackage{graphicx,grffile}
\makeatletter
\def\maxwidth{\ifdim\Gin@nat@width>\linewidth\linewidth\else\Gin@nat@width\fi}
\def\maxheight{\ifdim\Gin@nat@height>\textheight\textheight\else\Gin@nat@height\fi}
\makeatother
% Scale images if necessary, so that they will not overflow the page
% margins by default, and it is still possible to overwrite the defaults
% using explicit options in \includegraphics[width, height, ...]{}
\setkeys{Gin}{width=\maxwidth,height=\maxheight,keepaspectratio}
\IfFileExists{parskip.sty}{%
\usepackage{parskip}
}{% else
\setlength{\parindent}{0pt}
\setlength{\parskip}{6pt plus 2pt minus 1pt}
}
\setlength{\emergencystretch}{3em}  % prevent overfull lines
\providecommand{\tightlist}{%
  \setlength{\itemsep}{0pt}\setlength{\parskip}{0pt}}
\setcounter{secnumdepth}{0}
% Redefines (sub)paragraphs to behave more like sections
\ifx\paragraph\undefined\else
\let\oldparagraph\paragraph
\renewcommand{\paragraph}[1]{\oldparagraph{#1}\mbox{}}
\fi
\ifx\subparagraph\undefined\else
\let\oldsubparagraph\subparagraph
\renewcommand{\subparagraph}[1]{\oldsubparagraph{#1}\mbox{}}
\fi

%%% Use protect on footnotes to avoid problems with footnotes in titles
\let\rmarkdownfootnote\footnote%
\def\footnote{\protect\rmarkdownfootnote}

%%% Change title format to be more compact
\usepackage{titling}

% Create subtitle command for use in maketitle
\providecommand{\subtitle}[1]{
  \posttitle{
    \begin{center}\large#1\end{center}
    }
}

\setlength{\droptitle}{-2em}

  \title{EDA of us airquality}
    \pretitle{\vspace{\droptitle}\centering\huge}
  \posttitle{\par}
    \author{}
    \preauthor{}\postauthor{}
    \date{}
    \predate{}\postdate{}
  

\begin{document}
\maketitle

\subsection{prepare libraries}\label{prepare-libraries}

\begin{Shaded}
\begin{Highlighting}[]
\KeywordTok{library}\NormalTok{(ggplot2)}
\KeywordTok{library}\NormalTok{(tidyverse)}
\end{Highlighting}
\end{Shaded}

\begin{verbatim}
## -- Attaching packages --------------------------------------------------- tidyverse 1.2.1 --
\end{verbatim}

\begin{verbatim}
## v tibble  2.1.1       v purrr   0.3.2  
## v tidyr   0.8.3       v dplyr   0.8.0.1
## v readr   1.3.1       v stringr 1.3.1  
## v tibble  2.1.1       v forcats 0.4.0
\end{verbatim}

\begin{verbatim}
## -- Conflicts ------------------------------------------------------ tidyverse_conflicts() --
## x dplyr::filter() masks stats::filter()
## x dplyr::lag()    masks stats::lag()
\end{verbatim}

\subsection{load data from csv and view data
summary}\label{load-data-from-csv-and-view-data-summary}

\begin{quote}
limit:the ``annual mean, averaged over 3 years'' cannot exceed 12 μg/m3
\end{quote}

\begin{quote}
data source:
\url{https://raw.githubusercontent.com/DataScienceSpecialization/courses/master/04_ExploratoryAnalysis/exploratoryGraphs/PM25data.zip}
\href{https://www3.epa.gov/ttn/airs/airsaqs/detaildata/downloadaqsdata.htm}{EPAAirQualitySystem}
\end{quote}

\begin{Shaded}
\begin{Highlighting}[]
\NormalTok{airdata <-}\StringTok{ }\KeywordTok{read.csv}\NormalTok{(}\StringTok{'../data/avgpm25.csv'}\NormalTok{, }\DataTypeTok{colClasses =} \KeywordTok{c}\NormalTok{(}\StringTok{"numeric"}\NormalTok{, }\StringTok{"character"}\NormalTok{, }\StringTok{"factor"}\NormalTok{, }\StringTok{"numeric"}\NormalTok{, }\StringTok{"numeric"}\NormalTok{))}
\KeywordTok{head}\NormalTok{(airdata)}
\end{Highlighting}
\end{Shaded}

\begin{verbatim}
##        pm25  fips region longitude latitude
## 1  9.771185 01003   east -87.74826 30.59278
## 2  9.993817 01027   east -85.84286 33.26581
## 3 10.688618 01033   east -87.72596 34.73148
## 4 11.337424 01049   east -85.79892 34.45913
## 5 12.119764 01055   east -86.03212 34.01860
## 6 10.827805 01069   east -85.35039 31.18973
\end{verbatim}

\begin{Shaded}
\begin{Highlighting}[]
\KeywordTok{dim}\NormalTok{(airdata)}
\end{Highlighting}
\end{Shaded}

\begin{verbatim}
## [1] 576   5
\end{verbatim}

\begin{Shaded}
\begin{Highlighting}[]
\KeywordTok{summary}\NormalTok{(airdata)}
\end{Highlighting}
\end{Shaded}

\begin{verbatim}
##       pm25            fips            region      longitude      
##  Min.   : 3.383   Length:576         east:442   Min.   :-158.04  
##  1st Qu.: 8.549   Class :character   west:134   1st Qu.: -97.38  
##  Median :10.047   Mode  :character              Median : -87.37  
##  Mean   : 9.836                                 Mean   : -91.65  
##  3rd Qu.:11.356                                 3rd Qu.: -80.72  
##  Max.   :18.441                                 Max.   : -68.26  
##     latitude    
##  Min.   :19.68  
##  1st Qu.:35.30  
##  Median :39.09  
##  Mean   :38.56  
##  3rd Qu.:41.75  
##  Max.   :64.82
\end{verbatim}

\subsection{explore pm25}\label{explore-pm25}

\begin{itemize}
\tightlist
\item
  boxplot (single variable)
\end{itemize}

\begin{Shaded}
\begin{Highlighting}[]
\KeywordTok{qplot}\NormalTok{(}\StringTok{""}\NormalTok{, pm25, }\DataTypeTok{data =}\NormalTok{ airdata, }\DataTypeTok{geom=}\StringTok{"boxplot"}\NormalTok{, }\DataTypeTok{fill=}\KeywordTok{I}\NormalTok{(}\StringTok{"lightblue"}\NormalTok{), }\DataTypeTok{outlier.color =} \KeywordTok{I}\NormalTok{(}\StringTok{"red"}\NormalTok{))}
\end{Highlighting}
\end{Shaded}

\includegraphics{eda_us_airqulity_files/figure-latex/unnamed-chunk-4-1.pdf}

调整binwidth,可以看出离群点,

\begin{Shaded}
\begin{Highlighting}[]
\NormalTok{gg <-}\StringTok{ }\KeywordTok{ggplot}\NormalTok{(airdata, }\KeywordTok{aes}\NormalTok{(}\DataTypeTok{x=}\NormalTok{pm25))}
\NormalTok{gg }\OperatorTok{+}\StringTok{ }\KeywordTok{geom_histogram}\NormalTok{(}\DataTypeTok{binwidth =} \FloatTok{0.5}\NormalTok{)}
\end{Highlighting}
\end{Shaded}

\includegraphics{eda_us_airqulity_files/figure-latex/unnamed-chunk-5-1.pdf}

\textbf{叠加直方图和密度图} refer to {[}R graphics cookbook{]}

\begin{Shaded}
\begin{Highlighting}[]
\NormalTok{gg <-}\StringTok{ }\KeywordTok{ggplot}\NormalTok{(airdata, }\KeywordTok{aes}\NormalTok{(}\DataTypeTok{x=}\NormalTok{pm25,}\DataTypeTok{y=}\NormalTok{..density..))}
\NormalTok{gg }\OperatorTok{+}\StringTok{ }\KeywordTok{geom_histogram}\NormalTok{(}\DataTypeTok{binwidth =} \FloatTok{0.2}\NormalTok{) }\OperatorTok{+}\StringTok{ }\KeywordTok{geom_density}\NormalTok{(}\DataTypeTok{alpha=}\NormalTok{.}\DecValTok{2}\NormalTok{)}
\end{Highlighting}
\end{Shaded}

\includegraphics{eda_us_airqulity_files/figure-latex/unnamed-chunk-6-1.pdf}

\textbf{添加rug}

\begin{Shaded}
\begin{Highlighting}[]
\NormalTok{gg <-}\StringTok{ }\KeywordTok{ggplot}\NormalTok{(airdata, }\KeywordTok{aes}\NormalTok{(}\DataTypeTok{x=}\NormalTok{pm25))}
\NormalTok{gg  }\OperatorTok{+}\StringTok{ }\KeywordTok{geom_histogram}\NormalTok{(}\DataTypeTok{binwidth =} \FloatTok{0.2}\NormalTok{) }\OperatorTok{+}
\StringTok{  }\KeywordTok{geom_rug}\NormalTok{(}\DataTypeTok{sides =} \StringTok{"b"}\NormalTok{, }\DataTypeTok{alpha=}\NormalTok{.}\DecValTok{5}\NormalTok{, }\DataTypeTok{color=}\StringTok{"#0072B2"}\NormalTok{)}
\end{Highlighting}
\end{Shaded}

\includegraphics{eda_us_airqulity_files/figure-latex/unnamed-chunk-7-1.pdf}

\subsection{按region查看分布}\label{region}

\textbf{条形图}

\begin{Shaded}
\begin{Highlighting}[]
\NormalTok{gg =}\StringTok{ }\KeywordTok{ggplot}\NormalTok{(airdata, }\DataTypeTok{mapping =} \KeywordTok{aes}\NormalTok{(}\DataTypeTok{x=}\NormalTok{region))}

\NormalTok{gg }\OperatorTok{+}\StringTok{ }\KeywordTok{geom_bar}\NormalTok{() }\OperatorTok{+}\StringTok{ }\KeywordTok{ggtitle}\NormalTok{(}\StringTok{"Number of Countries in Each Region"}\NormalTok{)}
\end{Highlighting}
\end{Shaded}

\includegraphics{eda_us_airqulity_files/figure-latex/unnamed-chunk-8-1.pdf}

\texttt{结论}:东部的统计数量较多 \textbf{箱体图}

\begin{Shaded}
\begin{Highlighting}[]
\NormalTok{gg =}\StringTok{ }\KeywordTok{ggplot}\NormalTok{(airdata, }\DataTypeTok{mapping =} \KeywordTok{aes}\NormalTok{(}\DataTypeTok{x=}\NormalTok{region, }\DataTypeTok{y=}\NormalTok{pm25))}
\NormalTok{gg }\OperatorTok{+}\StringTok{ }\KeywordTok{geom_boxplot}\NormalTok{(}\DataTypeTok{fill =} \StringTok{"#0073C2FF"}\NormalTok{)}
\end{Highlighting}
\end{Shaded}

\includegraphics{eda_us_airqulity_files/figure-latex/unnamed-chunk-9-1.pdf}

\texttt{结论}: - 东部平均污染程度高,西部较低 - 东部个别地区很低,
西部个别地区很高。看离群点

\textbf{直方图}

\begin{Shaded}
\begin{Highlighting}[]
\NormalTok{airdata}\OperatorTok{$}\NormalTok{region <-}\StringTok{ }\KeywordTok{as.factor}\NormalTok{(airdata}\OperatorTok{$}\NormalTok{region)}
\NormalTok{gg =}\StringTok{ }\KeywordTok{ggplot}\NormalTok{(airdata, }\DataTypeTok{mapping =} \KeywordTok{aes}\NormalTok{(}\DataTypeTok{x=}\NormalTok{pm25))}
\NormalTok{gg }\OperatorTok{+}\StringTok{ }\KeywordTok{geom_histogram}\NormalTok{(}\DataTypeTok{fill =} \StringTok{"#0073C2FF"}\NormalTok{, }\DataTypeTok{binwidth =}\NormalTok{ .}\DecValTok{2}\NormalTok{) }\OperatorTok{+}\StringTok{ }\KeywordTok{facet_grid}\NormalTok{(region }\OperatorTok{~}\StringTok{ }\NormalTok{.) }\OperatorTok{+}\StringTok{ }\KeywordTok{geom_rug}\NormalTok{(}\DataTypeTok{sides =} \StringTok{"b"}\NormalTok{, }\DataTypeTok{alpha=}\NormalTok{.}\DecValTok{5}\NormalTok{, }\DataTypeTok{color=}\StringTok{"#0072B2"}\NormalTok{)}
\end{Highlighting}
\end{Shaded}

\includegraphics{eda_us_airqulity_files/figure-latex/unnamed-chunk-10-1.pdf}

\textbf{密度图}

\begin{Shaded}
\begin{Highlighting}[]
\NormalTok{airdata}\OperatorTok{$}\NormalTok{region <-}\StringTok{ }\KeywordTok{as.factor}\NormalTok{(airdata}\OperatorTok{$}\NormalTok{region)}
\NormalTok{gg =}\StringTok{ }\KeywordTok{ggplot}\NormalTok{(airdata, }\DataTypeTok{mapping =} \KeywordTok{aes}\NormalTok{(}\DataTypeTok{x=}\NormalTok{pm25))}
\NormalTok{gg }\OperatorTok{+}\StringTok{ }\KeywordTok{geom_density}\NormalTok{(}\DataTypeTok{fill =} \StringTok{"#0073C2FF"}\NormalTok{, }\DataTypeTok{binwidth =}\NormalTok{ .}\DecValTok{2}\NormalTok{) }\OperatorTok{+}\StringTok{ }\KeywordTok{facet_grid}\NormalTok{(region }\OperatorTok{~}\StringTok{ }\NormalTok{.) }\OperatorTok{+}\StringTok{ }\KeywordTok{geom_rug}\NormalTok{(}\DataTypeTok{sides =} \StringTok{"b"}\NormalTok{, }\DataTypeTok{alpha=}\NormalTok{.}\DecValTok{5}\NormalTok{, }\DataTypeTok{color=}\StringTok{"#0072B2"}\NormalTok{)}
\end{Highlighting}
\end{Shaded}

\begin{verbatim}
## Warning: Ignoring unknown parameters: binwidth
\end{verbatim}

\includegraphics{eda_us_airqulity_files/figure-latex/unnamed-chunk-11-1.pdf}


\end{document}
